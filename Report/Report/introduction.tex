% !TEX root = ../main.tex
% !TEX spellcheck = en_GB

\chapter{Introduction}
\label{sec:introduction}

\section{Project Description}
\label{sec:projectDescription}
A way of producing a great loud speaker, is to have a great model to test before the actual production. Our project will aim to create such a model, taking into account as many of the following variables:

\begin{itemize}
	\item Shape of speaker
	\item Size of speaker
	\item Number of units
	\item Size of units
	\item Placement of units
	\item Crossover filter
\end{itemize}

This model will be used to compare the frequency response of a known speaker, measured in the anechoic room, to the modelled speaker response.

If time is, an optimization for the flattest frequency response will be done within realistic bounds for the optimized values.


\section{Project delimitations}
\label{sec:delimitations}
The \nameref{sec:projectDescription} is the groups vision of the ideal solution to the problem outlined in the \nameref{sec:introduction}.  
To give the best possible view of the groups capabilities in developing such a solution, the most core parts of the project will have the highest priority, hence some parts will be excluded from this version of the project. 

\section{Terminology}
\label{sec:terminology}
Below in \cref{tab:terminology} is shown a list of terms used throughout the report describing each name for clarification purposes.

\begin{table}[H]
	\centering
	\begin{tabularx}{0.8\textwidth}{l X}
		\toprule
		\textbf{Name} & \textbf{Description} \\
		\midrule
		&\\
		\bottomrule
	\end{tabularx}
	\caption{List of terminologies.}
	\label{tab:terminology}
\end{table}